% Options for packages loaded elsewhere
\PassOptionsToPackage{unicode}{hyperref}
\PassOptionsToPackage{hyphens}{url}
\PassOptionsToPackage{dvipsnames,svgnames,x11names}{xcolor}
%
\documentclass[
  super,
  preprint,
  3p]{elsarticle}

\usepackage{amsmath,amssymb}
\usepackage{iftex}
\ifPDFTeX
  \usepackage[T1]{fontenc}
  \usepackage[utf8]{inputenc}
  \usepackage{textcomp} % provide euro and other symbols
\else % if luatex or xetex
  \usepackage{unicode-math}
  \defaultfontfeatures{Scale=MatchLowercase}
  \defaultfontfeatures[\rmfamily]{Ligatures=TeX,Scale=1}
\fi
\usepackage{lmodern}
\ifPDFTeX\else  
    % xetex/luatex font selection
\fi
% Use upquote if available, for straight quotes in verbatim environments
\IfFileExists{upquote.sty}{\usepackage{upquote}}{}
\IfFileExists{microtype.sty}{% use microtype if available
  \usepackage[]{microtype}
  \UseMicrotypeSet[protrusion]{basicmath} % disable protrusion for tt fonts
}{}
\makeatletter
\@ifundefined{KOMAClassName}{% if non-KOMA class
  \IfFileExists{parskip.sty}{%
    \usepackage{parskip}
  }{% else
    \setlength{\parindent}{0pt}
    \setlength{\parskip}{6pt plus 2pt minus 1pt}}
}{% if KOMA class
  \KOMAoptions{parskip=half}}
\makeatother
\usepackage{xcolor}
\setlength{\emergencystretch}{3em} % prevent overfull lines
\setcounter{secnumdepth}{5}
% Make \paragraph and \subparagraph free-standing
\ifx\paragraph\undefined\else
  \let\oldparagraph\paragraph
  \renewcommand{\paragraph}[1]{\oldparagraph{#1}\mbox{}}
\fi
\ifx\subparagraph\undefined\else
  \let\oldsubparagraph\subparagraph
  \renewcommand{\subparagraph}[1]{\oldsubparagraph{#1}\mbox{}}
\fi


\providecommand{\tightlist}{%
  \setlength{\itemsep}{0pt}\setlength{\parskip}{0pt}}\usepackage{longtable,booktabs,array}
\usepackage{calc} % for calculating minipage widths
% Correct order of tables after \paragraph or \subparagraph
\usepackage{etoolbox}
\makeatletter
\patchcmd\longtable{\par}{\if@noskipsec\mbox{}\fi\par}{}{}
\makeatother
% Allow footnotes in longtable head/foot
\IfFileExists{footnotehyper.sty}{\usepackage{footnotehyper}}{\usepackage{footnote}}
\makesavenoteenv{longtable}
\usepackage{graphicx}
\makeatletter
\def\maxwidth{\ifdim\Gin@nat@width>\linewidth\linewidth\else\Gin@nat@width\fi}
\def\maxheight{\ifdim\Gin@nat@height>\textheight\textheight\else\Gin@nat@height\fi}
\makeatother
% Scale images if necessary, so that they will not overflow the page
% margins by default, and it is still possible to overwrite the defaults
% using explicit options in \includegraphics[width, height, ...]{}
\setkeys{Gin}{width=\maxwidth,height=\maxheight,keepaspectratio}
% Set default figure placement to htbp
\makeatletter
\def\fps@figure{htbp}
\makeatother

\makeatletter
\@ifpackageloaded{caption}{}{\usepackage{caption}}
\AtBeginDocument{%
\ifdefined\contentsname
  \renewcommand*\contentsname{Table of contents}
\else
  \newcommand\contentsname{Table of contents}
\fi
\ifdefined\listfigurename
  \renewcommand*\listfigurename{List of Figures}
\else
  \newcommand\listfigurename{List of Figures}
\fi
\ifdefined\listtablename
  \renewcommand*\listtablename{List of Tables}
\else
  \newcommand\listtablename{List of Tables}
\fi
\ifdefined\figurename
  \renewcommand*\figurename{Figure}
\else
  \newcommand\figurename{Figure}
\fi
\ifdefined\tablename
  \renewcommand*\tablename{Table}
\else
  \newcommand\tablename{Table}
\fi
}
\@ifpackageloaded{float}{}{\usepackage{float}}
\floatstyle{ruled}
\@ifundefined{c@chapter}{\newfloat{codelisting}{h}{lop}}{\newfloat{codelisting}{h}{lop}[chapter]}
\floatname{codelisting}{Listing}
\newcommand*\listoflistings{\listof{codelisting}{List of Listings}}
\makeatother
\makeatletter
\makeatother
\makeatletter
\@ifpackageloaded{caption}{}{\usepackage{caption}}
\@ifpackageloaded{subcaption}{}{\usepackage{subcaption}}
\makeatother
\journal{Journal Name}
\ifLuaTeX
  \usepackage{selnolig}  % disable illegal ligatures
\fi
\usepackage[]{natbib}
\bibliographystyle{elsarticle-num}
\usepackage{bookmark}

\IfFileExists{xurl.sty}{\usepackage{xurl}}{} % add URL line breaks if available
\urlstyle{same} % disable monospaced font for URLs
\hypersetup{
  pdftitle={Redes Psicométricas para Dados Binários usando abordagem INLA},
  pdfauthor={Renato Rodrigues Silva; Márcio Augusto Ferreira Rodrigues; Everton Batista da Rocha},
  pdfkeywords={keyword1, keyword2},
  colorlinks=true,
  linkcolor={blue},
  filecolor={Maroon},
  citecolor={Blue},
  urlcolor={Blue},
  pdfcreator={LaTeX via pandoc}}

\setlength{\parindent}{6pt}
\begin{document}

\begin{frontmatter}
\title{Redes Psicométricas para Dados Binários usando abordagem INLA}
\author[1]{Renato Rodrigues Silva%
\corref{cor1}%
\fnref{fn1}}
 \ead{renato.rrsilva@ufg.br} 
\author[1]{Márcio Augusto Ferreira Rodrigues%
%
}
 \ead{marcioaugusto@ufg.br} 
\author[1]{Everton Batista da Rocha%
%
}
 \ead{evertonbatista@ufg.br} 

\affiliation[1]{organization={Federal University of Goias, Institute of
Mathematics and Statistics},addressline={Street
Address},city={Goiânia},postcode={74001-970},postcodesep={}}

\cortext[cor1]{Corresponding author}
\fntext[fn1]{This is the first author footnote.}


        
\begin{abstract}
This is the abstract. Lorem ipsum dolor sit amet, consectetur adipiscing
elit. Vestibulum augue turpis, dictum non malesuada a, volutpat eget
velit. Nam placerat turpis purus, eu tristique ex tincidunt et. Mauris
sed augue eget turpis ultrices tincidunt. Sed et mi in leo porta
egestas. Aliquam non laoreet velit. Nunc quis ex vitae eros aliquet
auctor nec ac libero. Duis laoreet sapien eu mi luctus, in bibendum leo
molestie. Sed hendrerit diam diam, ac dapibus nisl volutpat vitae.
Aliquam bibendum varius libero, eu efficitur justo rutrum at. Sed at
tempus elit.
\end{abstract}





\begin{keyword}
    keyword1 \sep 
    keyword2
\end{keyword}
\end{frontmatter}
    
\section{Introdução}\label{introduuxe7uxe3o}

Análise de rede psicométrica tem sido utilizada em diversas áreas do
conhecimento, tais como: psicologia ambiental, psicopatologia,
psicologia da personalidade, saúde pública entre outras \citep[
]{ZWICKER2020101433, Borsboom2017, COSTANTINI201513, Soares2022}. A
pressuposição dessa análise é que o fenômeno a ser estudado seja um
sistema complexo. Sendo assim, a análise de redes proporciona que a
representação das variáveis que compõem o sistema seja feita através de
um grafo não direcionado \citep{Murphy2012}. Além disso, é possível
fazer modelar as relações entre as variáveis e identificar padrões de
agrupamentos \citep{Epskamp2022}. Por exemplo, em uma aplicação
hipotética na área de psicopatologia, pode-se assumir que os constructos
de um transtorno de depressão maior e transtorno de ansiedade
generalizada poderiam formar um sistema complexo, composto por diversos
sintomas (fadiga, anedonia, humor hipotímico, insônia, irritabilidade,
dificuldade de concentração, tensão muscular, etc \ldots). E esses
sintomas se relacionam uns com os outros, de tal modo que eles se
retroalimentam mutuamente. Os sintomas seriam representados
matematicamente por nodos e a relação entre esses sintomas por arestas
de um grafo não direcionado. Nesse exemplo, o objetivo dos pesquisadores
seria verificar o padrão de agrupamentos dos sintomas e o relacionamento
entre eles.

A análise de redes psicométricas são baseadas nos modelos
probabilísticos denominados Campo Aleatório de Markov (CAM)
\citep{Murphy2012}. Nas situações em que se tem apenas variáveis
aleatórias contínuas, tem-se usado os modelos Campo Aleatório de Markov
Gaussianos. Por sua vez, para variáveis aleatórias dicotômicas, o modelo
CAM reduz-se aos modelos Ising \citep{Murphy2012}, objeto de interesse
desse presente estudo. Pode-se demostrar que o modelo Ising é
relacionado com o modelo de regressão logística, de tal forma que para
estimar a relação de associação entre um nodo e os demais, utiliza-se o
modelo logístico em que as covariáveis são os nodos remanescentes e a
variável resposta é o nodo em questão \citep{vanBorkulo}.

A estimação dos parâmetros desses modelos é algo desafiador. Por
exemplo, para o modelo com dados binários, o número de parâmetros é dado
por: \(\frac{k*(k+1)}{2},\) em que \(k\) é o número de nodos do grafo.
Sob o enfoque clássico ou frequentista, a solução encontrada foi estimar
os parâmetros \citep{Epskamp2018}.

Recentemente, observou-se publicações de estudos de análise de redes
psicométricas utilizando o enfoque Bayesiano

\citet{Marsman2023}

\citep{Huth, Marsman2023}.

\subsection{Campo Aleatório de
Markov}\label{campo-aleatuxf3rio-de-markov}

CAM pode ser definido como um conjunto de variáveis aleatórias

Ising Model

Relação entre Ising Model e Modelo Autologisitico

INLA

Aplicação

INLA Autologistic

bgms

Comparação de Resultados

Referências


  \bibliography{reference.bib}


\end{document}
