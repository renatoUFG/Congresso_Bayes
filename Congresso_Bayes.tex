% Options for packages loaded elsewhere
\PassOptionsToPackage{unicode}{hyperref}
\PassOptionsToPackage{hyphens}{url}
\PassOptionsToPackage{dvipsnames,svgnames,x11names}{xcolor}
%
\documentclass[
  super,
  preprint,
  3p]{elsarticle}

\usepackage{amsmath,amssymb}
\usepackage{iftex}
\ifPDFTeX
  \usepackage[T1]{fontenc}
  \usepackage[utf8]{inputenc}
  \usepackage{textcomp} % provide euro and other symbols
\else % if luatex or xetex
  \usepackage{unicode-math}
  \defaultfontfeatures{Scale=MatchLowercase}
  \defaultfontfeatures[\rmfamily]{Ligatures=TeX,Scale=1}
\fi
\usepackage{lmodern}
\ifPDFTeX\else  
    % xetex/luatex font selection
\fi
% Use upquote if available, for straight quotes in verbatim environments
\IfFileExists{upquote.sty}{\usepackage{upquote}}{}
\IfFileExists{microtype.sty}{% use microtype if available
  \usepackage[]{microtype}
  \UseMicrotypeSet[protrusion]{basicmath} % disable protrusion for tt fonts
}{}
\makeatletter
\@ifundefined{KOMAClassName}{% if non-KOMA class
  \IfFileExists{parskip.sty}{%
    \usepackage{parskip}
  }{% else
    \setlength{\parindent}{0pt}
    \setlength{\parskip}{6pt plus 2pt minus 1pt}}
}{% if KOMA class
  \KOMAoptions{parskip=half}}
\makeatother
\usepackage{xcolor}
\setlength{\emergencystretch}{3em} % prevent overfull lines
\setcounter{secnumdepth}{5}
% Make \paragraph and \subparagraph free-standing
\ifx\paragraph\undefined\else
  \let\oldparagraph\paragraph
  \renewcommand{\paragraph}[1]{\oldparagraph{#1}\mbox{}}
\fi
\ifx\subparagraph\undefined\else
  \let\oldsubparagraph\subparagraph
  \renewcommand{\subparagraph}[1]{\oldsubparagraph{#1}\mbox{}}
\fi


\providecommand{\tightlist}{%
  \setlength{\itemsep}{0pt}\setlength{\parskip}{0pt}}\usepackage{longtable,booktabs,array}
\usepackage{calc} % for calculating minipage widths
% Correct order of tables after \paragraph or \subparagraph
\usepackage{etoolbox}
\makeatletter
\patchcmd\longtable{\par}{\if@noskipsec\mbox{}\fi\par}{}{}
\makeatother
% Allow footnotes in longtable head/foot
\IfFileExists{footnotehyper.sty}{\usepackage{footnotehyper}}{\usepackage{footnote}}
\makesavenoteenv{longtable}
\usepackage{graphicx}
\makeatletter
\def\maxwidth{\ifdim\Gin@nat@width>\linewidth\linewidth\else\Gin@nat@width\fi}
\def\maxheight{\ifdim\Gin@nat@height>\textheight\textheight\else\Gin@nat@height\fi}
\makeatother
% Scale images if necessary, so that they will not overflow the page
% margins by default, and it is still possible to overwrite the defaults
% using explicit options in \includegraphics[width, height, ...]{}
\setkeys{Gin}{width=\maxwidth,height=\maxheight,keepaspectratio}
% Set default figure placement to htbp
\makeatletter
\def\fps@figure{htbp}
\makeatother

\makeatletter
\@ifpackageloaded{caption}{}{\usepackage{caption}}
\AtBeginDocument{%
\ifdefined\contentsname
  \renewcommand*\contentsname{Table of contents}
\else
  \newcommand\contentsname{Table of contents}
\fi
\ifdefined\listfigurename
  \renewcommand*\listfigurename{List of Figures}
\else
  \newcommand\listfigurename{List of Figures}
\fi
\ifdefined\listtablename
  \renewcommand*\listtablename{List of Tables}
\else
  \newcommand\listtablename{List of Tables}
\fi
\ifdefined\figurename
  \renewcommand*\figurename{Figure}
\else
  \newcommand\figurename{Figure}
\fi
\ifdefined\tablename
  \renewcommand*\tablename{Table}
\else
  \newcommand\tablename{Table}
\fi
}
\@ifpackageloaded{float}{}{\usepackage{float}}
\floatstyle{ruled}
\@ifundefined{c@chapter}{\newfloat{codelisting}{h}{lop}}{\newfloat{codelisting}{h}{lop}[chapter]}
\floatname{codelisting}{Listing}
\newcommand*\listoflistings{\listof{codelisting}{List of Listings}}
\makeatother
\makeatletter
\makeatother
\makeatletter
\@ifpackageloaded{caption}{}{\usepackage{caption}}
\@ifpackageloaded{subcaption}{}{\usepackage{subcaption}}
\makeatother
\journal{Journal Name}
\ifLuaTeX
  \usepackage{selnolig}  % disable illegal ligatures
\fi
\usepackage[]{natbib}
\bibliographystyle{elsarticle-num}
\usepackage{bookmark}

\IfFileExists{xurl.sty}{\usepackage{xurl}}{} % add URL line breaks if available
\urlstyle{same} % disable monospaced font for URLs
\hypersetup{
  pdftitle={Avaliação da Escolha de Hiperparâmetros para Modelos Gráficos de Cópula Bayesianos},
  pdfauthor={Renato Rodrigues Silva; Márcio Augusto Ferreira Rodrigues; Everton Batista da Rocha; Sandro Rogério Rodrigues Batista},
  pdfkeywords={keyword1, keyword2},
  colorlinks=true,
  linkcolor={blue},
  filecolor={Maroon},
  citecolor={Blue},
  urlcolor={Blue},
  pdfcreator={LaTeX via pandoc}}

\setlength{\parindent}{6pt}
\begin{document}

\begin{frontmatter}
\title{Avaliação da Escolha de Hiperparâmetros para Modelos Gráficos de
Cópula Bayesianos}
\author[1]{Renato Rodrigues Silva%
\corref{cor1}%
\fnref{fn1}}
 \ead{renato.rrsilva@ufg.br} 
\author[1]{Márcio Augusto Ferreira Rodrigues%
%
}
 \ead{marcioaugusto@ufg.br} 
\author[1]{Everton Batista da Rocha%
%
}
 \ead{evertonbatista@ufg.br} 
\author[2,3]{Sandro Rogério Rodrigues Batista%
%
}
 \ead{sandrorbatista@gmail.com} 

\affiliation[1]{organization={Federal University of Goias, Institute of
Mathematics and Statistics},addressline={Campus Samambaia,
CP},city={Goiânia},postcode={74001-970},postcodesep={}}
\affiliation[2]{organization={Federal University of Goias, Faculty of
Medicine},addressline={235 c/ 1a. s/n - S.
Universitário},city={Goiânia},postcode={74605-020},postcodesep={}}
\affiliation[3]{organization={University of Brasilia, Faculty of
Medicine},addressline={Campus Universitario Darcy Ribeiro Icc
Sul},city={Brasília},postcode={70910-900},postcodesep={}}

\cortext[cor1]{Corresponding author}
\fntext[fn1]{This is the first author footnote.}



        
\begin{abstract}
This is the abstract. Lorem ipsum dolor sit amet, consectetur adipiscing
elit. Vestibulum augue turpis, dictum non malesuada a, volutpat eget
velit. Nam placerat turpis purus, eu tristique ex tincidunt et. Mauris
sed augue eget turpis ultrices tincidunt. Sed et mi in leo porta
egestas. Aliquam non laoreet velit. Nunc quis ex vitae eros aliquet
auctor nec ac libero. Duis laoreet sapien eu mi luctus, in bibendum leo
molestie. Sed hendrerit diam diam, ac dapibus nisl volutpat vitae.
Aliquam bibendum varius libero, eu efficitur justo rutrum at. Sed at
tempus elit.
\end{abstract}





\begin{keyword}
    keyword1 \sep 
    keyword2
\end{keyword}
\end{frontmatter}
    
\section{Introdução}\label{introduuxe7uxe3o}

Análise de rede tem sido utilizada em diversas áreas do conhecimento,
tais como: psicologia ambiental, psicopatologia, psicologia da
personalidade, saúde pública entre outras \citep[
]{ZWICKER2020101433, Borsboom2017, COSTANTINI201513, Soares2022}. Com a
pressuposição de que o fenômeno a ser estudado seja um sistema complexo,
essa análise proporciona que a representação das variáveis que compõem o
sistema seja feita através de um grafo não direcionado
\citep{Murphy2012}. Além disso, é possível modelar as relações entre as
variáveis e identificar padrões de agrupamentos \citep{Epskamp2022}. Por
exemplo, em uma aplicação de saúde pública, o fenômeno de multimorbidade
crônica em uma população pode ser considerado um fenômeno complexo, cada
morbidade crônica (diabetes, pressão arterial, colesterol, depressão,
etc \ldots{} ) pode ser representada por um nodo e a relação entre essas
doenças crônicas por arestas de um grafo direcionado. A análise de rede
é um método estatístico baseadas em uma classe de modelos
probabilísticos denominados Campo Aleatório de Markov (CAM)
\citep{Murphy2012}. Para variáveis aleatórias contínuas, tem-se usado os
modelos Campo Aleatório de Markov Gaussianos \citep{Rue2005}. O processo
de estimação de parâmetros desse modelo é bastante desafiador, pois o
número de parâmetros é dado por: \(\frac{k*(k+1)}{2},\)em que \(k\) é o
número de nodos de um grafo não direcionado. Para lidar com o problema
de superajustamento, \citet{Epskamp2018b}, propuseram a estimação da
correlação parcial com penalização LASSO. Para a modelagem de dados
binários, \citet{vanBorkulo2014} propuseram um um modelo estatístico
para a representação da rede baseado no modelo Ising. Dada a
equivalência entre o modelo Ising e o modelo de regressão logistica, foi
proposto o uso da regressão logistíca com penalização LASSO
\citep{Friedman2010} para lidar com o problema de superajustamento.
\citet{JSSv093i08} apresentaram o modelo gráfico misto, o qual é um caso
especial da distribuição conjunta de um CAM que permite combinar um
conjunto arbitrário de distribuições condicionais que sejam membros da
família exponencial. Mais recentemente, \citet{Huth} apresentou um
pacote estatístico com diversos modelos gráficos Bayesianos, dentre eles
destaca-se o modelo gráfico de cópula Gaussiano \citep{JSSv089i03}, pois
este permite a modelagem tanto de dados contínuos, binários, ordinais e
mistos, independente dessa variável pertencer ou não a família
exponencial. Segundo \citet{Huth}, a vantagem de usar os modelos
Bayesianos seria a quantificação da incerteza do espaço de busca da
seleção dos modelos candidatos e a quantificação da precisão das
estimativas. No entanto, uma questão central ainda a ser respondida é
qual o impacto da escolha dos hiperparâmetros na distribuição a
posteriori do modelos gráfico de cópula Gaussiano. Dessa forma, o
presente estudo objetiva avaliar diferentes escolhas de hiperparâmetros
para a distribuição a priori e verificar qual impacto dessas escolhas na
precisão das estimativas de medidas de centralidade da rede. Com esta
finalidade, serão utilizados os dados de morbidades crônicas obtidos na
segunda onda do Estudo Longitudinal da Saúde dos Idosos Brasileiros
(ELSI-Brasil) \citep{Lima-Costa2018}

\section{Material e Método}\label{material-e-muxe9todo}

O conjunto de dados utilizado foi obtido a partir de registros de
autorelatos sobre morbidades crônicas dos participantes da segunda onda
do estudo longitudinal da saúde dos idosos brasileiros, ELSI
\citep{Lima-Costa2018}. Na etapa de processamento de dados, retirou-se
os dados faltantes, totalizando N observações.

\subsection{Campo Aleatório de
Markov}\label{campo-aleatuxf3rio-de-markov}

CAM pode ser definido como um conjunto de variáveis aleatórias

\subsection{Campo Aleatório de Markov
Gaussiano}\label{campo-aleatuxf3rio-de-markov-gaussiano}

\subsection{Modelos Gráficos de Cópula
Gaussiano}\label{modelos-gruxe1ficos-de-cuxf3pula-gaussiano}

\section{Resultados}\label{resultados}

\subsection{Avaliação do parâmetro de probabildidade de
inclusão}\label{avaliauxe7uxe3o-do-paruxe2metro-de-probabildidade-de-inclusuxe3o}


  \bibliography{reference.bib}


\end{document}
